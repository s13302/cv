\documentclass[]{friggeri-cv} % Add 'print' as an option into the square bracket to remove colors from this template for printing

\begin{document}

	\header{Rafał}{Podkoński}{Senior Developer} % Your name and current job title/field
	
	\begin{aside}
		%----------------------------------------------------------------------------------------
		%	CONTACT
		%----------------------------------------------------------------------------------------
		\section{Kontakt}
		Wiktorów
		ul. Stołeczna 96
		05-083 Zaborów
		Polska
		~
		\phone{(+48) 511-548-637}
		\email{rafal.podkonski@dev-software.pl}
		~
		\linkedin{rafał-podkoński}{952339137}
		\facebook{rafal.podkonski}
		\github{s13302}
		
		%----------------------------------------------------------------------------------------
		%	LANGUAGES
		%----------------------------------------------------------------------------------------
		\section{Języki}
		\textbf{Polski} - Ojczysty
		\textbf{Angielski} - Zaawansowany
		\textbf{Hiszpański} - Podstawowy
		\textbf{Rosyjski} - Podstawowy
		\textbf{Niderlandzki} - Podstawowy
		%----------------------------------------------------------------------------------------
		%	PROGRAMMING LANGUAGES
		%----------------------------------------------------------------------------------------
		\section{Programowanie}
		\textbf{Backend:}
		Java, Spring Framework, Node.JS, Socket.IO, ExpressJS, PHP
		\textbf{Frontend:}
		HTML5, CSS3, JavaScript, Angular JS, Angular 2+, ReactJS, Vue.js
		\textbf{Bazy Danych:}
		PostgreSQL, Oracle, MSSQL, MongoDB
		\textbf{Inne:}
		\LaTeX
	\end{aside}
	
	%----------------------------------------------------------------------------------------
	%	WORK SUMMARY
	%----------------------------------------------------------------------------------------
	\begin{absolutelynopagebreak}
		\section{Podsumowanie Zawodowe}
		Jestem doświadczonym programistą Java, który interesuje się zagadnieniami związanymi z architekturą oprogramowania. Chciałbym poszerzać swoją wiedzę oraz doświadczenie w zakresie Domain Driven Design.
	\end{absolutelynopagebreak}
	
	%----------------------------------------------------------------------------------------
	%	WORK EXPERIENCE
	%----------------------------------------------------------------------------------------
	\begin{absolutelynopagebreak}
		\section{Doświadczenie}
		\begin{entrylist}
			%------------------------------------------------
			\entry
			{07.2017-obecnie}
			{Sollers Consulting Sp. z o.o.}
			{ul. Koszykowa 54, Warszawa}
			{\emph{Senior Developer}\\
				Implementacja procesów biznesowych w platformie Guidewire dla klientów z całego świata. Dostosowanie wtyczki IntelliJ Sonar do obsługi języka Gosu.
			}
			%------------------------------------------------
			\entry
			{03.2015-obecnie}
			{Koło Naukowe Programistów Java}
			{ul. Koszykowa 86, Warszawa}
			{\emph{Przewodniczący, Mentor}\\
				Zarządzanie kołem naukowym, zarządzanie zespołem mentorów. Organizacja wykładów i warsztatów związanych z programowaniem. Prowadzenie zajęć dla grup podstawowych i zaawansowanych z zakresu programowania w języku Java.
			}
			%------------------------------------------------
			\entry
			{06.2014-07.2017}
			{Innotion Sp. z o.o.}
			{ul. Łucka 2/4/6, Warszawa}
			{\emph{Developer}\\
				Tworzenie nowych rozwiązań, utrzymanie istniejących aplikacji dla klientów zewnętrznych.
			}
		\end{entrylist}
	\end{absolutelynopagebreak}
	
	%----------------------------------------------------------------------------------------
	%	EDUCATION
	%----------------------------------------------------------------------------------------
	\begin{absolutelynopagebreak}
		\section{Edukacja}
		\begin{entrylist}
			%------------------------------------------------
			\entry
			{2014-obecnie}
			{Polsko-Japońska Akademia Technik Komputerowych}
			{Koszykowa 86, Warszawa}
			{\emph{Kierunek:} Inżynierskie Informatyka\\
			\emph{Specjalizacja:} Programowanie Aplikacji Biznesowych}
			%------------------------------------------------
			\entry
			{2011-2014}
			{Uniwersytet im. Kardynała Stefana Wyszyńskiego}
			{Wóycickiego 1/3, Warszawa}
			{\emph{Kierunek:} Licencjackie Informatyka}
			%------------------------------------------------
			\entry
			{2007-2011}
			{Technikum Mechatroniczne nr 1 w Warszawie}
			{Wiśniowa 56, Warszawa}
			{\emph{Kierunek:} Technik Informatyk\\
			\emph{Specjalizacja:} Systemy i Sieci Komputerowe wg. programu CISCO CCNA}
		\end{entrylist}
	\end{absolutelynopagebreak}
	
	%----------------------------------------------------------------------------------------
	%	CERTIFICATES
	%----------------------------------------------------------------------------------------
	\begin{absolutelynopagebreak}
		\section{Certyfikaty i Uprawnienia}
		\begin{entrylist}
			%------------------------------------------------
			\cert
			{01.2019}
			{Guidewire InsuranceSuite Integration 10.0}
			%------------------------------------------------
			\cert
			{12.2018}
			{Guidewire InsuranceSuite Integration 9.0}
			%------------------------------------------------
			\cert
			{08.2017}
			{Guidewire PolicyCenter 9.0 Configuration}
			%------------------------------------------------
			\cert
			{05.2009}
			{Certyfikat ECDL}
			%------------------------------------------------
			\cert
			{04.2010}
			{Prawo jazdy kat. B}
		\end{entrylist}
	\end{absolutelynopagebreak}
	
	%----------------------------------------------------------------------------------------
	%	PROJECTS
	%----------------------------------------------------------------------------------------
	\begin{absolutelynopagebreak}
		\section{Projekty}
		\begin{entrylist}
			%------------------------------------------------
			\entry
			{04.2020-obecnie}
			{System rejestracji i obsługi zgłoszeń w OTOZ Animals}
			{Java EE, Spring Framework, Spring Data, Hibernate, PostgreSQL, Angular 6}
			{System wspierający rejestrację oraz obsługę zgłoszeń. System pozwala na rejestrację wydanych zaleceń, znalezionych zwierząt oraz ich stanu.}
			%------------------------------------------------
			\entry
			{04.2020-obecnie}
			{System głosowania dla ESN}
			{Node.JS, ExpressJS, Socket.IO, ReactJS, PostgreSQL, MongoDB, RWD}
			{System wspierający głosowanie ESN, kluczowym wymaganiem systemu była możliwość pracy realtime oraz możliwość pracy na smartfonie bez instalacji zewnętrznego oprogramowania.}
			%------------------------------------------------
			\entry
			{06.2016-07.2016}
			{System rejestracji grup pielgrzymkowych do pociągów oraz do muzeów}
			{Java EE, Spring, Spring Data, Angular JS}
			{System wspierający rejestrację grup pielgrzymów do muzeów oraz na pociągi. System wykonany na zlecenie Archidiecezji Warszawskiej.}
			%------------------------------------------------
			\entry
			{02.2013-04.2013}
			{Parser eksportów z programu Agencja 3000}
			{PHP, PostgreSQL}
			{Parser eksportu w formacie XML, dane ładowane do bazy danych PostgreSQL, a następnie na stronę WWW agencji nieruchomości.}
		\end{entrylist}
	\end{absolutelynopagebreak}
	
	%----------------------------------------------------------------------------------------
	%	COURSES
	%----------------------------------------------------------------------------------------
	\begin{absolutelynopagebreak}
		\section{Kursy}
		\begin{entrylist}
			%------------------------------------------------
			\cert
			{2019}
			{Droga Nowoczesnego Architekta}
			%------------------------------------------------
			\cert
			{2011}
			{CISCO CCNA}
		\end{entrylist}
	\end{absolutelynopagebreak}
	
	%----------------------------------------------------------------------------------------
	%	INTERESTS
	%----------------------------------------------------------------------------------------
	\begin{absolutelynopagebreak}
		\section{Zainteresowania}
		\textbf{Zawodowe: } tworzenie stron WWW, pisanie oprogramowania, architektura aplikacji, Domain Driven Design, Test Driven Design, Scrum\\
		\textbf{Osobiste: } jazda na rowerze, hokej na lodzie, tenis
	\end{absolutelynopagebreak}
	
	\vfill
	\textit{Wyrażam zgodę na przetwarzanie moich danych osobowych zawartych w mojej ofercie pracy dla potrzeb niezbędnych do realizacji procesu rekrutacji zgodnie z Ustawą z dn. 29.08.1997 roku o ochronie danych osobowych. Przysługuje mi prawo dostępu do treści swoich danych oraz ich poprawiania.}

\end{document}